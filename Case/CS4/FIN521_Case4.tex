%%%%%%%%%%%%%%%%%%%%%%%%%%%%%%%%%%%%%%%%%
% Structured General Purpose Assignment
% LaTeX Template
%
% This template has been downloaded from:
% http://www.latextemplates.com
%
% Original author:
% Ted Pavlic (http://www.tedpavlic.com)
%
% Note:
% The \lipsum[#] commands throughout this template generate dummy text
% to fill the template out. These commands should all be removed when
% writing assignment content.
%
%%%%%%%%%%%%%%%%%%%%%%%%%%%%%%%%%%%%%%%%%

%----------------------------------------------------------------------------------------
%	PACKAGES AND OTHER DOCUMENT CONFIGURATIONS
%----------------------------------------------------------------------------------------

\documentclass{article}

\usepackage{fancyhdr} % Required for custom headers
\usepackage{lastpage} % Required to determine the last page for the footer
\usepackage{extramarks} % Required for headers and footers
\usepackage{graphicx} % Required to insert images
\usepackage{lipsum} % Used for inserting dummy 'Lorem ipsum' text into the template
\usepackage{enumerate}
\usepackage{booktabs}
\usepackage{amsmath}
\usepackage{booktabs}
\usepackage{multirow}

% Margins
\topmargin=-0.45in
\evensidemargin=0in
\oddsidemargin=0in
\textwidth=6.5in
\textheight=9.0in
\headsep=0.25in

\linespread{1.5} % Line spacing

% Set up the header and footer
\pagestyle{fancy}
\lhead{\hmwkAuthorName} % Top left header
\chead{\hmwkClass\ (\hmwkTitle)} % Top center header
%%\rhead{\firstxmark}
\rhead{} % Top right header
\lfoot{\lastxmark} % Bottom left footer
\cfoot{} % Bottom center footer
\rfoot{Page\ \thepage\ of\ \pageref{LastPage}} % Bottom right footer
\renewcommand\headrulewidth{0.4pt} % Size of the header rule
\renewcommand\footrulewidth{0.4pt} % Size of the footer rule

\setlength\parindent{0pt} % Removes all indentation from paragraphs

%----------------------------------------------------------------------------------------
%	DOCUMENT STRUCTURE COMMANDS
%	Skip this unless you know what you're doing
%----------------------------------------------------------------------------------------

% Header and footer for when a page split occurs within a problem environment
\newcommand{\enterProblemHeader}[1]{
\nobreak\extramarks{#1}{#1 continued on next page\ldots}\nobreak
\nobreak\extramarks{#1 (continued)}{#1 continued on next page\ldots}\nobreak
}

% Header and footer for when a page split occurs between problem environments
\newcommand{\exitProblemHeader}[1]{
\nobreak\extramarks{#1 (continued)}{#1 continued on next page\ldots}\nobreak
\nobreak\extramarks{#1}{}\nobreak
}

\setcounter{secnumdepth}{0} % Removes default section numbers
\newcounter{homeworkProblemCounter} % Creates a counter to keep track of the number of problems

\newcommand{\homeworkProblemName}{}
\newenvironment{homeworkProblem}[1][Question \arabic{homeworkProblemCounter}]{ % Makes a new environment called homeworkProblem which takes 1 argument (custom name) but the default is "Question #"
\stepcounter{homeworkProblemCounter} % Increase counter for number of problems
\renewcommand{\homeworkProblemName}{#1} % Assign \homeworkProblemName the name of the problem
\section{\homeworkProblemName} % Make a section in the document with the custom problem count
\enterProblemHeader{\homeworkProblemName} % Header and footer within the environment
}{
\exitProblemHeader{\homeworkProblemName} % Header and footer after the environment
}

\newcommand{\problemAnswer}[1]{ % Defines the problem answer command with the content as the only argument
\noindent\framebox[\columnwidth][c]{\begin{minipage}{0.98\columnwidth}#1\end{minipage}} % Makes the box around the problem answer and puts the content inside
}

\newcommand{\homeworkSectionName}{}
\newenvironment{homeworkSection}[1]{ % New environment for sections within homework problems, takes 1 argument - the name of the section
\renewcommand{\homeworkSectionName}{#1} % Assign \homeworkSectionName to the name of the section from the environment argument
\subsection{\homeworkSectionName} % Make a subsection with the custom name of the subsection
\enterProblemHeader{\homeworkProblemName\ [\homeworkSectionName]} % Header and footer within the environment
}{
\enterProblemHeader{\homeworkProblemName} % Header and footer after the environment
}

%----------------------------------------------------------------------------------------
%	EXPECTATION AND VARIANCE OPERATOR
%----------------------------------------------------------------------------------------
 \newcommand{\E}{\mathrm{E}}
 \newcommand{\Var}{\mathrm{Var}}
 \newcommand{\Cov}{\mathrm{Cov}}
 \newcommand{\Corr}{\mathrm{Corr}}

%----------------------------------------------------------------------------------------
%	NAME AND CLASS SECTION
%----------------------------------------------------------------------------------------

\newcommand{\hmwkTitle}{Case 4: Blaine Kitchenware} % Assignment title
\newcommand{\hmwkDueDate}{Tuesday,\ April\ 3,\ 2018} % Due date
\newcommand{\hmwkClass}{FIN\ 521} % Course/class
\newcommand{\hmwkClassTime}{2:00pm} % Class/lecture time
\newcommand{\hmwkAuthorName}{Wanbae Park} % Your name

%----------------------------------------------------------------------------------------
%	TITLE PAGE
%----------------------------------------------------------------------------------------

\title{
\vspace{2in}
\textmd{\textbf{\hmwkClass:\ \hmwkTitle}}\\
\normalsize\vspace{0.1in}\small{Due\ on\ \hmwkDueDate}\\
\vspace{3in}
}

\author{\textbf{\hmwkAuthorName}}
\date{} % Insert date here if you want it to appear below your name

%----------------------------------------------------------------------------------------

\begin{document}

\maketitle

%----------------------------------------------------------------------------------------
%	TABLE OF CONTENTS
%----------------------------------------------------------------------------------------

%\setcounter{tocdepth}{1} % Uncomment this line if you don't want subsections listed in the ToC

%%\newpage
%%\tableofcontents
\newpage

%----------------------------------------------------------------------------------------
%	Question 1
%----------------------------------------------------------------------------------------
\begin{homeworkProblem}

\end{homeworkProblem}

%----------------------------------------------------------------------------------------
%	Question 2
%----------------------------------------------------------------------------------------
\begin{homeworkProblem}
    The current capital structure and payout policies of the firm
    are not appropriate since there are lots of benefits from using debt.
    Of course, increasing leverage makes firm's revenue risky,
    but on the current status, compensation from increasing leverage is much larger
    since the current amount of debt is zero.
    Most of firms which uses debt has larger market-to-book ratio,
    and larger ROE than Blaine Kitchenware. Assuming that industrial background
    among the firms are negligible, it justifies the argument that Blaine should
    raise debt. It is reasonable to increase D/V ratio to 17.09\%, which is
    the average D/V ratio of other firms.
\end{homeworkProblem}

%----------------------------------------------------------------------------------------
%	Question 3
%----------------------------------------------------------------------------------------
\begin{homeworkProblem}
    \begin{enumerate}[a)]
        \item   %% question a)
        \begin{enumerate}[1)]
            \item   %% 1)
            From exhibit 3 of the case, we can find cash, enterprise value,
            debt, and equity of the firm at 2006.
            Table \ref{tab:bs_before_announcement} shows the balance sheet
            before announcement.
            \begin{table}[ht]
\centering
\begin{tabular}{@{}cccc@{}}
\toprule
\multicolumn{2}{c}{Asset}  & \multicolumn{2}{c}{Liabilities} \\ \midrule
Cash             & 230,866 & Debt           & 0              \\
Enterprise Value & 728,730 & Equity         & 959,596        \\ \bottomrule
\end{tabular}
\caption{Balance sheet before announcement}
\label{tab:bs_before_announcement}
\end{table}

            \item   %% 2)
            After the announcement but before any debt issuance or repurchase
            is made, under perfect market, the balance sheet does not change
            because the announcement itself does not affect capital structure
            and systematic risk of the firm. Therefore, the balance sheet
            after announcement is same as Table \ref{tab:bs_before_announcement}.
            \item   %% 3)
            After the firm issue 100,000(thousand dollars) amount of debt,
            debt increases and cash also increases on balance sheet.
            Therefore, the balance sheet of the firm changes to
            Table \ref{tab:bs_after_debt_perfect}. Equity value and enterprise
            value does not change.
            \begin{table}[ht]
\centering
\begin{tabular}{@{}cccc@{}}
\toprule
\multicolumn{2}{c}{Asset}  & \multicolumn{2}{c}{Liabilities} \\ \midrule
Cash             & 330,866 & Debt           & 100,000              \\
Enterprise Value & 728,730 & Equity         & 959,596        \\ \bottomrule
\end{tabular}
\caption{Balance sheet after issuing debt(Perfect market)}
\label{tab:bs_after_debt_perfect}
\end{table}

            \item   %% 4)
            After repurchase, 300,000 amount of cash decreases, and the same
            value of equity also decreases. Although market value of equity
            decreased, because the same amount of cash also decreased,
            therefore, enterprise value does not change. Furthermore, since
            number of shares also decreased, stock price also does not change.
            \begin{table}[ht]
\centering
\begin{tabular}{@{}cccc@{}}
\toprule
\multicolumn{2}{c}{Asset}  & \multicolumn{2}{c}{Liabilities} \\ \midrule
Cash             & 30,866 & Debt           & 100,000              \\
Enterprise Value & 728,730 & Equity         & 659,596        \\ \bottomrule
\end{tabular}
\caption{Balance sheet after repurchase(Perfect market)}
\label{tab:bs_after_repurchase_perfect}
\end{table}

        \end{enumerate}
        \item   %% question b)
        \begin{enumerate}[1)]
            \item   %% 1)
            Before the announcement, the balance sheet is same as
            Table \ref{bs_before_announcement}.
            \item   %% 2)
            
            \item   %% 3)
            \item   %% 4)
        \end{enumerate}
    \end{enumerate}
\end{homeworkProblem}

%----------------------------------------------------------------------------------------
%	Question 4
%----------------------------------------------------------------------------------------
\begin{homeworkProblem}
    \begin{enumerate}[a)]
        \item   %% question a)
        \item   %% question b)
        \item   %% question c)
        \item   %% question d)
        \item   %% question e)
        \item   %% question f)
        \item   %% question g)
    \end{enumerate}
\end{homeworkProblem}

%----------------------------------------------------------------------------------------
%	Question 5
%----------------------------------------------------------------------------------------
\begin{homeworkProblem}
    \begin{enumerate}[a)]
        \item   %% question a)
        \item   %% question b)
    \end{enumerate}
\end{homeworkProblem}

%----------------------------------------------------------------------------------------
%	Question 6
%----------------------------------------------------------------------------------------
\begin{homeworkProblem}

\end{homeworkProblem}


\end{document}
