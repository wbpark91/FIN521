%%%%%%%%%%%%%%%%%%%%%%%%%%%%%%%%%%%%%%%%%
% Structured General Purpose Assignment
% LaTeX Template
%
% This template has been downloaded from:
% http://www.latextemplates.com
%
% Original author:
% Ted Pavlic (http://www.tedpavlic.com)
%
% Note:
% The \lipsum[#] commands throughout this template generate dummy text
% to fill the template out. These commands should all be removed when
% writing assignment content.
%
%%%%%%%%%%%%%%%%%%%%%%%%%%%%%%%%%%%%%%%%%

%----------------------------------------------------------------------------------------
%	PACKAGES AND OTHER DOCUMENT CONFIGURATIONS
%----------------------------------------------------------------------------------------

\documentclass{article}

\usepackage{fancyhdr} % Required for custom headers
\usepackage{lastpage} % Required to determine the last page for the footer
\usepackage{extramarks} % Required for headers and footers
\usepackage{graphicx} % Required to insert images
\usepackage{lipsum} % Used for inserting dummy 'Lorem ipsum' text into the template
\usepackage{enumerate}
\usepackage{booktabs}
\usepackage{amsmath}
\usepackage{booktabs}
\usepackage{multirow}
\usepackage[normalem]{ulem}
\useunder{\uline}{\ul}{}

% Margins
\topmargin=-0.45in
\evensidemargin=0in
\oddsidemargin=0in
\textwidth=6.5in
\textheight=9.0in
\headsep=0.25in

\linespread{1.5} % Line spacing

% Set up the header and footer
\pagestyle{fancy}
% \lhead{\hmwkAuthorName} % Top left header
\chead{\hmwkClass\ (\hmwkTitle)} % Top center header
%%\rhead{\firstxmark}
\rhead{} % Top right header
\lfoot{\lastxmark} % Bottom left footer
\cfoot{} % Bottom center footer
\rfoot{Page\ \thepage\ of\ \pageref{LastPage}} % Bottom right footer
\renewcommand\headrulewidth{0.4pt} % Size of the header rule
\renewcommand\footrulewidth{0.4pt} % Size of the footer rule

\setlength\parindent{0pt} % Removes all indentation from paragraphs

%----------------------------------------------------------------------------------------
%	DOCUMENT STRUCTURE COMMANDS
%	Skip this unless you know what you're doing
%----------------------------------------------------------------------------------------

% Header and footer for when a page split occurs within a problem environment
\newcommand{\enterProblemHeader}[1]{
\nobreak\extramarks{#1}{#1 continued on next page\ldots}\nobreak
\nobreak\extramarks{#1 (continued)}{#1 continued on next page\ldots}\nobreak
}

% Header and footer for when a page split occurs between problem environments
\newcommand{\exitProblemHeader}[1]{
\nobreak\extramarks{#1 (continued)}{#1 continued on next page\ldots}\nobreak
\nobreak\extramarks{#1}{}\nobreak
}

\setcounter{secnumdepth}{0} % Removes default section numbers
\newcounter{homeworkProblemCounter} % Creates a counter to keep track of the number of problems

\newcommand{\homeworkProblemName}{}
\newenvironment{homeworkProblem}[1][Question \arabic{homeworkProblemCounter}]{ % Makes a new environment called homeworkProblem which takes 1 argument (custom name) but the default is "Question #"
\stepcounter{homeworkProblemCounter} % Increase counter for number of problems
\renewcommand{\homeworkProblemName}{#1} % Assign \homeworkProblemName the name of the problem
\section{\homeworkProblemName} % Make a section in the document with the custom problem count
\enterProblemHeader{\homeworkProblemName} % Header and footer within the environment
}{
\exitProblemHeader{\homeworkProblemName} % Header and footer after the environment
}

\newcommand{\problemAnswer}[1]{ % Defines the problem answer command with the content as the only argument
\noindent\framebox[\columnwidth][c]{\begin{minipage}{0.98\columnwidth}#1\end{minipage}} % Makes the box around the problem answer and puts the content inside
}

\newcommand{\homeworkSectionName}{}
\newenvironment{homeworkSection}[1]{ % New environment for sections within homework problems, takes 1 argument - the name of the section
\renewcommand{\homeworkSectionName}{#1} % Assign \homeworkSectionName to the name of the section from the environment argument
\subsection{\homeworkSectionName} % Make a subsection with the custom name of the subsection
\enterProblemHeader{\homeworkProblemName\ [\homeworkSectionName]} % Header and footer within the environment
}{
\enterProblemHeader{\homeworkProblemName} % Header and footer after the environment
}

%----------------------------------------------------------------------------------------
%	EXPECTATION AND VARIANCE OPERATOR
%----------------------------------------------------------------------------------------
 \newcommand{\E}{\mathrm{E}}
 \newcommand{\Var}{\mathrm{Var}}
 \newcommand{\Cov}{\mathrm{Cov}}
 \newcommand{\Corr}{\mathrm{Corr}}

%----------------------------------------------------------------------------------------
%	NAME AND CLASS SECTION
%----------------------------------------------------------------------------------------

\newcommand{\hmwkTitle}{Case 4: Blaine Kitchenware} % Assignment title
\newcommand{\hmwkDueDate}{Tuesday,\ April\ 3,\ 2018} % Due date
\newcommand{\hmwkClass}{FIN\ 521} % Course/class
\newcommand{\hmwkClassTime}{2:00pm} % Class/lecture time
\newcommand{\hmwkAuthorName}{Wanbae Park, Sangwoo Park, Inhyuk Lee, Soyeon Chang} % Your name

%----------------------------------------------------------------------------------------
%	TITLE PAGE
%----------------------------------------------------------------------------------------

\title{
\vspace{2in}
\textmd{\textbf{\hmwkClass:\ \hmwkTitle}}\\
\normalsize\vspace{0.1in}\small{Due\ on\ \hmwkDueDate}\\
\vspace{3in}
}

\author{\textbf{\hmwkAuthorName}}
\date{} % Insert date here if you want it to appear below your name

%----------------------------------------------------------------------------------------

\begin{document}

\maketitle

%----------------------------------------------------------------------------------------
%	TABLE OF CONTENTS
%----------------------------------------------------------------------------------------

%\setcounter{tocdepth}{1} % Uncomment this line if you don't want subsections listed in the ToC

%%\newpage
%%\tableofcontents
\newpage

%----------------------------------------------------------------------------------------
%	Question 1
%----------------------------------------------------------------------------------------
\begin{homeworkProblem}
    Payout policy and capital structure affect firm value because market
    is not perfect. Under M\&M, they are irrelevant to firm value since
    they does not affect future cash flow and systematic risk of the firm.
    However, in reality, there exists corporate tax, agency cost,
    and cost of financial distress.
    They make payout policy and capital structure are relevant to firm value.
\end{homeworkProblem}

%----------------------------------------------------------------------------------------
%	Question 2
%----------------------------------------------------------------------------------------
\begin{homeworkProblem}
    The current capital structure and payout policies of the firm
    are not appropriate since there are lots of benefits from using debt.
    Of course, increasing leverage makes firm's revenue risky,
    but on the current status, compensation from increasing leverage is much larger
    since the current amount of debt is zero.
    Most of firms which uses debt has larger market-to-book ratio,
    and larger ROE than Blaine Kitchenware. Assuming that industrial background
    among the firms are negligible, it justifies the argument that Blaine should
    raise debt. It is reasonable to increase D/V ratio to 17.09\%, which is
    the average D/V ratio of other firms.
    Considering kitchenware industry is in steady state, payout ratio over
    50\% might not be too large. However, considering the firm's strategy:
    acquiring independent manufacturer to expand business, it is needed to
    retain certain amount of earnings.
\end{homeworkProblem}

%----------------------------------------------------------------------------------------
%	Question 3
%----------------------------------------------------------------------------------------
\begin{homeworkProblem}
    \begin{enumerate}[a)]
        \item   %% question a)
        \begin{enumerate}[1)]
            \item   %% 1)
            From exhibit 3 of the case, we can find cash, enterprise value,
            debt, and equity of the firm at 2006.
            Table \ref{tab:bs_before_announcement} shows the balance sheet
            before announcement.
            \begin{table}[ht]
\centering
\begin{tabular}{@{}cccc@{}}
\toprule
\multicolumn{2}{c}{Asset}  & \multicolumn{2}{c}{Liabilities} \\ \midrule
Cash             & 230,866 & Debt           & 0              \\
Enterprise Value & 728,730 & Equity         & 959,596        \\ \bottomrule
\end{tabular}
\caption{Balance sheet before announcement}
\label{tab:bs_before_announcement}
\end{table}

            \item   %% 2)
            After the announcement but before any debt issuance or repurchase
            is made, under perfect market, the balance sheet does not change
            because the announcement itself does not affect capital structure
            and systematic risk of the firm. Therefore, the balance sheet
            after announcement is same as Table \ref{tab:bs_before_announcement}.
            \item   %% 3)
            After the firm issue 100,000(thousand dollars) amount of debt,
            debt increases and cash also increases on balance sheet.
            Therefore, the balance sheet of the firm changes to
            Table \ref{tab:bs_after_debt_perfect}. Equity value and enterprise
            value does not change.
            \begin{table}[ht]
\centering
\begin{tabular}{@{}cccc@{}}
\toprule
\multicolumn{2}{c}{Asset}  & \multicolumn{2}{c}{Liabilities} \\ \midrule
Cash             & 330,866 & Debt           & 100,000              \\
Enterprise Value & 728,730 & Equity         & 959,596        \\ \bottomrule
\end{tabular}
\caption{Balance sheet after issuing debt(Perfect market)}
\label{tab:bs_after_debt_perfect}
\end{table}

            \item   %% 4)
            After repurchase, 300,000 amount of cash decreases, and the same
            value of equity also decreases. Although market value of equity
            decreased, because the same amount of cash also decreased,
            therefore, enterprise value does not change. Furthermore, since
            number of shares also decreased, stock price also does not change.
            \begin{table}[ht]
\centering
\begin{tabular}{@{}cccc@{}}
\toprule
\multicolumn{2}{c}{Asset}  & \multicolumn{2}{c}{Liabilities} \\ \midrule
Cash             & 30,866 & Debt           & 100,000              \\
Enterprise Value & 728,730 & Equity         & 659,596        \\ \bottomrule
\end{tabular}
\caption{Balance sheet after repurchase(Perfect market)}
\label{tab:bs_after_repurchase_perfect}
\end{table}

            Stock price before repurchase is \$16.25 per share, so the number
            of share repurchased is equal to $300,000 / 16.25 = 18,461$, and
            the number outstanding share is decreased to $59,052 - 18,461 = 40,591$.
            Hence, the new share price is equal to $659,596 / 40,591 = 16.25$,
            which is same as the price before repurchase.
        \end{enumerate}
        \item   %% question b)
        \begin{enumerate}[1)]
            \item   %% 1)
            Before the announcement, the balance sheet is same as
            Table \ref{tab:bs_before_announcement}.
            \item   %% 2)
            Unlike answers in question a)-2), the balance sheet does change
            because there is tax benefit if debt is issued and cash is decreased,
            and equity value reflects the benefit. Increase in equity value
            would be same as present value of tax benefit from issuing debt
            and decrease in cash.
            Present value of tax shield from debt is calculated by multiplying
            interest expense to corporate tax rate, and discounting at cost of debt.
            Assuming Blaine's credit rating is Baa, corporate bond yield is
            equal to 6.72\%. Therefore, interest tax shield at each fiscal year is
            calculated as $100,000 \times 6.72\% \times 30.76\% = 2,067.072$.
            Therefore, present value of income tax shield from issuing debt
            is equal to $\frac{2,067.072}{0.06} = 34,451.2$.
            Assuming amount of cash on balance sheet grows at risk-free rate,
            if the firm does not repurchase stock, it should pay
            $200,000 \times 5.10\% \times 30.76\% = 3,137.52$ for tax every year.
            (We used 30-year US treasury rate as risk-free rate.) Therefore,
            by using perpetuity formula, the present value of tax benefit from
            cash deducting is equal to $\frac{3,137.52}{0.06} = 52,292$.
            Hence, after the announcement of stock repurchase, equity value
            would increase by $34,451.2 + 52,292 = 86,743.2$. Since other
            accounts are constant, enterprise value would also increase by
            86,743.2 thousand dollars. Table \ref{tab:bs_after_announcement_imperfect}
            represents balance sheet after the announcement under market in which
            corporate tax exists.
            \begin{table}[ht]
\centering
\begin{tabular}{@{}cccc@{}}
\toprule
\multicolumn{2}{c}{Asset}  & \multicolumn{2}{c}{Liabilities} \\ \midrule
Cash             & 230,866 & Debt           & 0              \\
Enterprise Value & 815,473.2 & Equity         & 1,046,339.2        \\ \bottomrule
\end{tabular}
\caption{Balance sheet after announcement(Imperfect market)}
\label{tab:bs_after_announcement_imperfect}
\end{table}

            Since the number of outstanding share in 2006 is 59,052 and it
            remains constant after the announcement, stock price per share
            would increase to $1,046,339.2 / 59,052 = \$17.72$.
            \item   %% 3)
            Like question a)-3), issuance of debt increases both debt and cash
            by 100,000. Table \ref{tab:bs_after_debt_imperfect} represents
            the balance sheet after issuance of debt. It does not affect enterprise
            value.
            \begin{table}[ht]
\centering
\begin{tabular}{@{}cccc@{}}
\toprule
\multicolumn{2}{c}{Asset}  & \multicolumn{2}{c}{Liabilities} \\ \midrule
Cash             & 330,866 & Debt           & 100,000              \\
Enterprise Value & 815,473.2 & Equity         & 1,046,339.2        \\ \bottomrule
\end{tabular}
\caption{Balance sheet after issuing debt(Imperfect market)}
\label{tab:bs_after_debt_imperfect}
\end{table}

            \item   %% 4)
            The repurchase of stock makes cash and equity decrease by 300,000.
            Table \ref{tab:bs_after_repurchase_imperfect} represents the result
            after repurchase of shares. The repurchase does not make stock price
            change because the information of repurchase was already reflected
            before the actual repurchase.
            \begin{table}[ht]
\centering
\begin{tabular}{@{}cccc@{}}
\toprule
\multicolumn{2}{c}{Asset}  & \multicolumn{2}{c}{Liabilities} \\ \midrule
Cash             & 30,866 & Debt           & 100,000              \\
Enterprise Value & 815,473.2 & Equity         & 746,339.2        \\ \bottomrule
\end{tabular}
\caption{Balance sheet after repurchase(Imperfect market)}
\label{tab:bs_after_repurchase_imperfect}
\end{table}

            The amount of share repurchased is equal to $300,000 / 17.72 = 16,931$.
            Therefore, the number of outstanding share decreases to 42,121.
            Since the new equity value is equal to 746,339, the new share price
            is equal to $746,339 / 42,121 = 17.72$, which is same as the price
            before repurchase.
        \end{enumerate}
    \end{enumerate}
\end{homeworkProblem}

%----------------------------------------------------------------------------------------
%	Question 4
%----------------------------------------------------------------------------------------
\begin{homeworkProblem}
    \begin{enumerate}[a)]
        \item   %% question a)
        Table \ref{tab:q4-bs} shows balance sheet before and after repurchase.
        The * mark indicates accounts which changes after repurchase.
        We assumed that the firm uses cash first and liquidates marketable
        securities to repurchase shares. Since cash and marketable securities
        decreases, current asset and total asset also decreases.
        On liabilities, long-term debt increases by 100,000 by the issuance, and
        shareholder's equity decreases by 300,000 by repurchase.
        \begin{table}[ht]
\centering
\begin{tabular}{lcc}
\toprule
\textit{Assets:}                              & 2006             & 2006(New)        \\  \midrule
Cash \& Cash Equivalents*                     & 66,557           & 0                \\
Marketable Securities*                        & 164,309          & 30,866           \\
Accounts Receivable                           & 48,780           & 48,780           \\
Inventory                                     & 54,874           & 54,874           \\
Other Current Assets                          & 5,157            & 5,157            \\
\textbf{Total Current Assets}                 & \textbf{339,678} & \textbf{139,678} \\
                                              &                  &                  \\
Property, Plant \& Equipment                  & 174,321          & 174,321          \\
Goodwill                                      & 38,281           & 38,281           \\
Other Assets                                  & 39,973           & 39,973           \\
\textbf{Total Assets}                         & \textbf{592,253} & \textbf{392,253} \\
                                              &                  &                  \\
\textit{Liabilities \& Shareholders' Equity:} &                  &                  \\
Accounts Payable                              & 31,936           & 31,936           \\
Accrued Liabilities                           & 27,761           & 27,761           \\
Taxes Payable                                 & 16,884           & 16,884           \\
\textbf{Total Current Liabilities}            & \textbf{76,581}  & \textbf{76,581}  \\
                                              &                  &                  \\
Long-Term Debt*                               &                  & 100,000          \\
Other liabilities                             & 4,814            & 4,814            \\
Deferred Taxes                                & 22,495           & 22,495           \\
Total Liabilities                             & 103,890          & 203,890          \\
Shareholders' Equity*                         & 488,363          & 188,363          \\
\textbf{Total Liabilities \& Shareholders' Equity}     & \textbf{592,253}    & \textbf{392,253} \\ \bottomrule
\end{tabular}
\caption{Balance sheet before and after repurchase}
\label{tab:q4-bs}
\end{table}

        Table \ref{tab:q4-is} shows income statement before and after repurchase.
        Since cash and marketable securities decreases, other income(which is
        interest income from its position) also decreases. Since debt is issued,
        interest expense is added to the income statement. Consequently, because
        EBT changes, income tax also changes, and net income changes.
        \begin{table}[ht]
\centering
\begin{tabular}{@{}lcc@{}}
\toprule
\textit{Operating Results:}               & 2006          & 2006(New)     \\ \midrule
\textbf{Revenue}                          & 342,251       & 342,251       \\
Less:  Cost of Goods Sold                 & {\ul 249,794} & {\ul 249,794} \\
\textbf{Gross Profit}                     & 92,458        & 92,458        \\
Less:  Selling, General \& Administrative & {\ul 28,512}  & {\ul 28,512}  \\
\textbf{Operating Income}                 & 63,946        & 63,946        \\
Plus:  Depreciation \& Amortization       & {\ul 9,914}   & {\ul 9,914}   \\
\textbf{EBITDA}                           & 73,860        & 73,860        \\
                                          &               &               \\
\textbf{EBIT}                             & 63,946        & 63,946        \\
Plus:  Other Income (expense)             & 13,506        & 1,806         \\
Less: Interest Expense*                   &               & {\ul 6,720}   \\
\textbf{Earnings Before Tax}              & 77,451        & 59,031        \\
Less:  Taxes*                             & {\ul 23,821}  & {\ul 18,156}  \\
\textbf{Net Income}                       & 53,630        & 40,876        \\
Dividends                                 & 28,345        & 28,345        \\ \bottomrule
\end{tabular}
\caption{Income statement before and after repurchase}
\label{tab:q4-is}
\end{table}

        \item   %% question b)
        Table \ref{tab:q4-eps_roe_mtb} represents EPS, ROE, and market to book
        ratio before and after repurchase. All of three increases if the firm
        repurchases stocks.
        \begin{table}[ht]
\centering
\begin{tabular}{@{}lcc@{}}
\toprule
                              & 2006          & 2006(New)     \\ \midrule
Net Income                    & 53,630        & 40,876        \\
Number of share               & 59,052        & 42,121        \\
\textbf{EPS}                  & \textbf{0.91} & \textbf{0.97} \\
                              &               &               \\
Net Income                    & 53,630        & 40,876        \\
Book Value of Equity          & 488,363       & 188,363       \\
\textbf{ROE}                  & \textbf{0.11} & \textbf{0.22} \\
                              &               &               \\
Market Value of Equity        & 959,596       & 746,339       \\
Book Value of Equity          & 488,363       & 188,363       \\
\textbf{Market-to-Book Ratio} & \textbf{1.96} & \textbf{3.96} \\ \bottomrule
\end{tabular}
\caption{EPS, ROE, Market-to-Book ratio}
\label{tab:q4-eps_roe_mtb}
\end{table}

        \item   %% question c)
        Table \ref{tab:q4-intcov_dtoe} represents interest coverage and
        debt to equity ratio of the firm. Since there is no interest expense
        before repurchase, interest coverage is infinity. After repurchase,
        interest coverage becomes about 10.99. Regarding this value,
        the firm is able to cover interest expense. Debt to equity ratio becomes
        -0.47 to 0.37.
        \begin{table}[ht]
\centering
\begin{tabular}{lcc}
\toprule
                           & 2006            & 2006(New)      \\    \midrule
EBITDA                     & 73,860          & 73,860         \\
Interest Expense           & 0               & 6,720          \\
\textbf{Interest Coverage} & {$\mathbf\infty$}       & \textbf{10.99} \\
                           &                 &                \\
Net Debt                   & (230,866)       & 69,134         \\
Book Value of Equity       & 488,363         & 188,363        \\
\textbf{Debt/Equity}       & \textbf{(0.47)} & \textbf{0.37}  \\    \bottomrule
\end{tabular}
\caption{Interest Coverage, Debt to Equity}
\label{tab:q4-intcov_dtoe}
\end{table}

        \item   %% question d)
        Before repurchase, the founding family's ownership share is equal to
        62\% of total outstanding share, which is about 36,612 shares. After
        repurchase, since the total outstanding share decreases to 42,121,
        proportion of family's ownership increases to 87\%.
        \item   %% question e)
        Table \ref{tab:q4-totalcf} shows total cash flow for investors.
        The total cash flow decreases since interest income from cash and
        marketable securities decreases. The amount of decrease in cash flow
        is equal to 6,035. However, by repurchasing shares, price per share
        increases, therefore equity investors can get capital gain from increase
        of share price.
        \begin{table}[ht]
\centering
\begin{tabular}{@{}lll@{}}
\toprule
                 & 2006            & 2006(New)       \\ \midrule
Net Income       & 53,630          & 40,876          \\
Interest Expense & 0               & 6,720           \\
\textbf{Total}   & \textbf{53,630} & \textbf{47,596} \\ \bottomrule
\end{tabular}
\caption{Total cash flow}
\label{tab:q4-totalcf}
\end{table}

        \item   %% question f)
        Table \ref{tab:q4-ev} shows equity value, share price, market-to-book,
        and enterprise value of the firm. After repurchase, equity value
        decreases, but the amount of decrease is less than 300,000.
        Although equity value decreases, since number of outstanding share
        also decreases, in result, price per share increases.
        Market-to-book ratio increases from 1.96 to 3.96, and enterprise value
        increases by about 90,000.
        \begin{table}[ht]
\centering
\begin{tabular}{@{}lcc@{}}
\toprule
                              & 2006             & 2006(New)        \\ \midrule
\textbf{Price}                & \textbf{16.25}   & \textbf{17.72}   \\
Outstanding Shares            & 59,052           & 42,121           \\
\textbf{Market Value of Equity}        & \textbf{959,596}          & \textbf{746,339}          \\
Net Debt                      & (230,866)        & 69,134           \\
\textbf{EnterPrise Value}     & \textbf{728,730} & \textbf{815,473} \\
                              &                  &                  \\
Market Value of Equity        & 959,596          & 746,339          \\
Book Value of Equity          & 488,363          & 188,363          \\
\textbf{Market-to-Book Ratio} & \textbf{1.96}    & \textbf{3.96}    \\ \bottomrule
\end{tabular}
\caption{Equity value, share price, market-to-book, enterprise value}
\label{tab:q4-ev}
\end{table}

        \item   %% question g)
        From e), total cash flow decreases after repurchase. Of course, it is
        negative effect of repurchase, but since enterprise value increases
        much more than decrease in cash flow, it is better to repurchase share.
    \end{enumerate}
\end{homeworkProblem}

%----------------------------------------------------------------------------------------
%	Question 5
%----------------------------------------------------------------------------------------
\begin{homeworkProblem}
    \begin{enumerate}[a)]
        \item   %% question a)
        If the firm pays dividend instead of stock repurchase,
        ignoring investor-level tax difference between capital gain
        and dividend, the only thing which changes is number of outstanding
        shares. Unlike stock repurchase, number of shares remains constant
        to 59,052 if the firm pays dividend.
        Since number of shares is different, price per share and EPS will
        also be different. Price per share will be decline to
        $746,339 / 59,052 = 12.64$, and EPS will be decline to
        $40,876 / 59,052 = 0.69$. Others will not change.
        \item   %% question b)
        If the firm repurchase 12 million shares with price \$25 per share,
        number of outstanding share will be $59,052 - 12,000 = 47,052$.
        Therefore, price per share will be $746,339 / 47,052 = 15.86$,
        and EPS will be $40,876 / 47,052 = 0.87$.
    \end{enumerate}
\end{homeworkProblem}

%----------------------------------------------------------------------------------------
%	Question 6
%----------------------------------------------------------------------------------------
\begin{homeworkProblem}
    Dubinski should recommend repurchase to Blaine's board since there are
    lots of advantage from share repurchase.
    The main advantage of share repurchase is increase in enterprise value
    and price per share because of tax shield.
    However, considering the firm's strategy: acquiring small independent
    manufacturers, repurchase share using cash would make future cost of capital.
\end{homeworkProblem}
\end{document}
