%%%%%%%%%%%%%%%%%%%%%%%%%%%%%%%%%%%%%%%%%
% Structured General Purpose Assignment
% LaTeX Template
%
% This template has been downloaded from:
% http://www.latextemplates.com
%
% Original author:
% Ted Pavlic (http://www.tedpavlic.com)
%
% Note:
% The \lipsum[#] commands throughout this template generate dummy text
% to fill the template out. These commands should all be removed when
% writing assignment content.
%
%%%%%%%%%%%%%%%%%%%%%%%%%%%%%%%%%%%%%%%%%

%----------------------------------------------------------------------------------------
%	PACKAGES AND OTHER DOCUMENT CONFIGURATIONS
%----------------------------------------------------------------------------------------

\documentclass{article}

\usepackage{fancyhdr} % Required for custom headers
\usepackage{lastpage} % Required to determine the last page for the footer
\usepackage{extramarks} % Required for headers and footers
\usepackage{graphicx} % Required to insert images
\usepackage{lipsum} % Used for inserting dummy 'Lorem ipsum' text into the template
\usepackage{enumerate}
\usepackage{booktabs}
\usepackage{amsmath}
\usepackage{booktabs}
\usepackage{multirow}

% Margins
\topmargin=-0.45in
\evensidemargin=0in
\oddsidemargin=0in
\textwidth=6.5in
\textheight=9.0in
\headsep=0.25in

\linespread{1.5} % Line spacing

% Set up the header and footer
\pagestyle{fancy}
\lhead{\hmwkAuthorName} % Top left header
\chead{\hmwkClass\ (\hmwkTitle)} % Top center header
%%\rhead{\firstxmark}
\rhead{} % Top right header
\lfoot{\lastxmark} % Bottom left footer
\cfoot{} % Bottom center footer
\rfoot{Page\ \thepage\ of\ \pageref{LastPage}} % Bottom right footer
\renewcommand\headrulewidth{0.4pt} % Size of the header rule
\renewcommand\footrulewidth{0.4pt} % Size of the footer rule

\setlength\parindent{0pt} % Removes all indentation from paragraphs

%----------------------------------------------------------------------------------------
%	DOCUMENT STRUCTURE COMMANDS
%	Skip this unless you know what you're doing
%----------------------------------------------------------------------------------------

% Header and footer for when a page split occurs within a problem environment
\newcommand{\enterProblemHeader}[1]{
\nobreak\extramarks{#1}{#1 continued on next page\ldots}\nobreak
\nobreak\extramarks{#1 (continued)}{#1 continued on next page\ldots}\nobreak
}

% Header and footer for when a page split occurs between problem environments
\newcommand{\exitProblemHeader}[1]{
\nobreak\extramarks{#1 (continued)}{#1 continued on next page\ldots}\nobreak
\nobreak\extramarks{#1}{}\nobreak
}

\setcounter{secnumdepth}{0} % Removes default section numbers
\newcounter{homeworkProblemCounter} % Creates a counter to keep track of the number of problems

\newcommand{\homeworkProblemName}{}
\newenvironment{homeworkProblem}[1][Part \arabic{homeworkProblemCounter}]{ % Makes a new environment called homeworkProblem which takes 1 argument (custom name) but the default is "Question #"
\stepcounter{homeworkProblemCounter} % Increase counter for number of problems
\renewcommand{\homeworkProblemName}{#1} % Assign \homeworkProblemName the name of the problem
\section{\homeworkProblemName} % Make a section in the document with the custom problem count
\enterProblemHeader{\homeworkProblemName} % Header and footer within the environment
}{
\exitProblemHeader{\homeworkProblemName} % Header and footer after the environment
}

\newcommand{\problemAnswer}[1]{ % Defines the problem answer command with the content as the only argument
\noindent\framebox[\columnwidth][c]{\begin{minipage}{0.98\columnwidth}#1\end{minipage}} % Makes the box around the problem answer and puts the content inside
}

\newcommand{\homeworkSectionName}{}
\newenvironment{homeworkSection}[1]{ % New environment for sections within homework problems, takes 1 argument - the name of the section
\renewcommand{\homeworkSectionName}{#1} % Assign \homeworkSectionName to the name of the section from the environment argument
\subsection{\homeworkSectionName} % Make a subsection with the custom name of the subsection
\enterProblemHeader{\homeworkProblemName\ [\homeworkSectionName]} % Header and footer within the environment
}{
\enterProblemHeader{\homeworkProblemName} % Header and footer after the environment
}

%----------------------------------------------------------------------------------------
%	EXPECTATION AND VARIANCE OPERATOR
%----------------------------------------------------------------------------------------
 \newcommand{\E}{\mathrm{E}}
 \newcommand{\Var}{\mathrm{Var}}
 \newcommand{\Cov}{\mathrm{Cov}}
 \newcommand{\Corr}{\mathrm{Corr}}

%----------------------------------------------------------------------------------------
%	NAME AND CLASS SECTION
%----------------------------------------------------------------------------------------

\newcommand{\hmwkTitle}{MIL Exercise} % Assignment title
\newcommand{\hmwkDueDate}{Friday,\ March\ 30,\ 2018} % Due date
\newcommand{\hmwkClass}{FIN\ 521} % Course/class
\newcommand{\hmwkClassTime}{2:00pm} % Class/lecture time
\newcommand{\hmwkAuthorName}{Wanbae Park} % Your name

%----------------------------------------------------------------------------------------
%	TITLE PAGE
%----------------------------------------------------------------------------------------

\title{
\vspace{2in}
\textmd{\textbf{\hmwkClass:\ \hmwkTitle}}\\
\normalsize\vspace{0.1in}\small{Due\ on\ \hmwkDueDate}\\
\vspace{3in}
}

\author{\textbf{\hmwkAuthorName}}
\date{} % Insert date here if you want it to appear below your name

%----------------------------------------------------------------------------------------

\begin{document}

\maketitle

%----------------------------------------------------------------------------------------
%	TABLE OF CONTENTS
%----------------------------------------------------------------------------------------

%\setcounter{tocdepth}{1} % Uncomment this line if you don't want subsections listed in the ToC

%%\newpage
%%\tableofcontents
\newpage

%----------------------------------------------------------------------------------------
%	PART 1
%----------------------------------------------------------------------------------------
\begin{homeworkProblem}
    I chose PEPSICO INC. which is famous multinational food, snack, and
    beverage corporation. Most of data used in evaluating share price is come
    from CapitalIQ, and if data which is necessary to evaluate stock price
    is unobserved, it is estimated by myself.
    \begin{enumerate}[a)]
        \item   %% question a)
        I collected analyst's forecast of sales revenue of PepsiCo over next
        seven years. Table \ref{tab:part1-expectedcf} shows the estimated
        revenue. Estimated revenue grows moderately.
        \item   %% question b)
        Analyst's forecast data of company's EBITDA from FY2018 to FY2022
        is collected from CapitalIQ, and since the average EBITDA growth is
        about 6\% per year, I assumed that EBITDA will grow at 6\% over the
        last 2 years. Table \ref{tab:part1-expectedcf} shows the estimated
        EBITDA. Using this data, EBITDA/Sales margin was calculated.
        Table \ref{tab:part1-margins} shows estimated EBITDA/Sales margin.
        \item   %% question c)
        In order to estimate free cash flow, estimation for EBIT is needed first.
        As mentioned above, EBITDA was estimated by using analyst's forecast
        and by estimation of growth. Depreciation and amortization is estimated
        by estimating depreciation margin which is (Depreciation / Sales Revenue).
        Depreciation margin is estimated by averaging past 3 years' margin, and
        is assumed constant over 7 years. The margin was calculatd as 3.44\%.
        Changes in net working capital is estimated by estimating days in
        receivables, inventory, payables and others. They were estimated by
        averaging past 3 years' data and were assumed to be constant over 7 years.
        Table \ref{tab:part1-estimated days} shows estimated days of each account.
        Capital expenditure is estimated by estimating CapEx margin, which was
        calculated by averaging past 3 year's data. From the procedure, free
        cash flow was calculated as Table \ref{tab:part1-expectedcf}.
        \begin{table}[ht]
\centering
\begin{tabular}{@{}cccccccc@{}}
\toprule
                           & FY2018  & FY2019  & FY2020  & FY2021  & FY2022  & FY2023  & FY2024  \\ \midrule
Revenues                   & 65,373  & 67,472  & 69,619  & 71,229  & 73,403  & 77,621  & 80,373  \\
Gross Profit               & 35,782  & 36,931  & 38,105  & 38,987  & 40,176  & 42,485  & 43,992  \\
EBITDA                     & 13,627  & 14,402  & 15,129  & 16,286  & 17,188  & 18,219  & 19,312  \\
Depreciation/Amortization  & 2,248   & 2,321   & 2,394   & 2,450   & 2,525   & 2,670   & 2,764   \\
Operating Income           & 11,378  & 12,081  & 12,735  & 13,836  & 14,663  & 15,550  & 16,548  \\
Taxes                      & 2,389   & 2,537   & 2,674   & 2,906   & 3,079   & 3,265   & 3,475   \\
Tax Affected EBIT          & 8,989   & 9,544   & 10,060  & 10,931  & 11,584  & 12,284  & 13,073  \\ \midrule
Depreciation/Amortization  & 2,248   & 2,321   & 2,394   & 2,450   & 2,525   & 2,670   & 2,764   \\
Changes in Working Capital & 382     & (6)     & (6)     & (5)     & (6)     & (12)    & (8)     \\
Capital Expenditures       & (3,026) & (3,124) & (3,223) & (3,298) & (3,398) & (3,593) & (3,721) \\ \midrule
Unlevered Free Cash Flow   & 8,593   & 8,735   & 9,226   & 10,078  & 10,704  & 11,348  & 12,109  \\ \bottomrule
\end{tabular}
\caption{Estimated Income Statement and Expected Cash Flow}
\label{tab:part1-expectedcf}
\end{table}

        \begin{table}[ht]
\centering
\begin{tabular}{@{}cccccccc@{}}
\toprule
                    & FY2018  & FY2019  & FY2020  & FY2021  & FY2022  & FY2023  & FY2024  \\ \midrule
Sales Growth        & 2.91\%  & 3.21\%  & 3.18\%  & 2.31\%  & 3.05\%  & 5.75\%  & 3.55\%  \\
Gross Margin        & 54.73\% & 54.73\% & 54.73\% & 54.73\% & 54.73\% & 54.73\% & 54.73\% \\
EBITDA Margin       & 20.84\% & 21.34\% & 21.73\% & 22.86\% & 23.42\% & 23.47\% & 24.03\% \\
Depreciation Margin & 3.44\%  & 3.44\%  & 3.44\%  & 3.44\%  & 3.44\%  & 3.44\%  & 3.44\%  \\
CapEx Margin        & 4.63\%  & 4.63\%  & 4.63\%  & 4.63\%  & 4.63\%  & 4.63\%  & 4.63\%  \\ \bottomrule
\end{tabular}
\caption{Estimated Margins}
\label{tab:part1-margins}
\end{table}

        \begin{table}[ht]
\centering
\begin{tabular}{@{}ccccc@{}}
\toprule
                  & FY2015 & FY2016 & FY2017 & Average \\ \midrule
Days in AR        & 37     & 39     & 40     & 39      \\
Days in Inventory & 35     & 35     & 37     & 36      \\
Prepaid Days      & 14     & 11     & 19     & 19      \\
Days in AP        & 70     & 80     & 85     & 85      \\
Days in Accruals  & 52     & 55     & 53     & 53      \\ \bottomrule
\end{tabular}
\caption{Estimated days}
\label{tab:part1-estimated days}
\end{table}

        \item   %% question d)
        \item   %% question e)
        \item   %% question f)
        \item   %% question g)
        \item   %% question h)
    \end{enumerate}
\end{homeworkProblem}

%----------------------------------------------------------------------------------------
%	PART 2
%----------------------------------------------------------------------------------------
\begin{homeworkProblem}
    \begin{enumerate}[a)]
        \item   %% question a)
        \item   %% question b)
    \end{enumerate}
\end{homeworkProblem}

%----------------------------------------------------------------------------------------
%	PART 3
%----------------------------------------------------------------------------------------
\begin{homeworkProblem}

\end{homeworkProblem}

%----------------------------------------------------------------------------------------
%	PART 4
%----------------------------------------------------------------------------------------
\begin{homeworkProblem}

\end{homeworkProblem}

\end{document}
