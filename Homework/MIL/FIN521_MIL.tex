%%%%%%%%%%%%%%%%%%%%%%%%%%%%%%%%%%%%%%%%%
% Structured General Purpose Assignment
% LaTeX Template
%
% This template has been downloaded from:
% http://www.latextemplates.com
%
% Original author:
% Ted Pavlic (http://www.tedpavlic.com)
%
% Note:
% The \lipsum[#] commands throughout this template generate dummy text
% to fill the template out. These commands should all be removed when
% writing assignment content.
%
%%%%%%%%%%%%%%%%%%%%%%%%%%%%%%%%%%%%%%%%%

%----------------------------------------------------------------------------------------
%	PACKAGES AND OTHER DOCUMENT CONFIGURATIONS
%----------------------------------------------------------------------------------------

\documentclass{article}

\usepackage{fancyhdr} % Required for custom headers
\usepackage{lastpage} % Required to determine the last page for the footer
\usepackage{extramarks} % Required for headers and footers
\usepackage{graphicx} % Required to insert images
\usepackage{lipsum} % Used for inserting dummy 'Lorem ipsum' text into the template
\usepackage{enumerate}
\usepackage{booktabs}
\usepackage{amsmath}
\usepackage{booktabs}
\usepackage{multirow}

% Margins
\topmargin=-0.45in
\evensidemargin=0in
\oddsidemargin=0in
\textwidth=6.5in
\textheight=9.0in
\headsep=0.25in

\linespread{1.5} % Line spacing

% Set up the header and footer
\pagestyle{fancy}
\lhead{\hmwkAuthorName} % Top left header
\chead{\hmwkClass\ (\hmwkTitle)} % Top center header
%%\rhead{\firstxmark}
\rhead{} % Top right header
\lfoot{\lastxmark} % Bottom left footer
\cfoot{} % Bottom center footer
\rfoot{Page\ \thepage\ of\ \pageref{LastPage}} % Bottom right footer
\renewcommand\headrulewidth{0.4pt} % Size of the header rule
\renewcommand\footrulewidth{0.4pt} % Size of the footer rule

\setlength\parindent{0pt} % Removes all indentation from paragraphs

%----------------------------------------------------------------------------------------
%	DOCUMENT STRUCTURE COMMANDS
%	Skip this unless you know what you're doing
%----------------------------------------------------------------------------------------

% Header and footer for when a page split occurs within a problem environment
\newcommand{\enterProblemHeader}[1]{
\nobreak\extramarks{#1}{#1 continued on next page\ldots}\nobreak
\nobreak\extramarks{#1 (continued)}{#1 continued on next page\ldots}\nobreak
}

% Header and footer for when a page split occurs between problem environments
\newcommand{\exitProblemHeader}[1]{
\nobreak\extramarks{#1 (continued)}{#1 continued on next page\ldots}\nobreak
\nobreak\extramarks{#1}{}\nobreak
}

\setcounter{secnumdepth}{0} % Removes default section numbers
\newcounter{homeworkProblemCounter} % Creates a counter to keep track of the number of problems

\newcommand{\homeworkProblemName}{}
\newenvironment{homeworkProblem}[1][Part \arabic{homeworkProblemCounter}]{ % Makes a new environment called homeworkProblem which takes 1 argument (custom name) but the default is "Question #"
\stepcounter{homeworkProblemCounter} % Increase counter for number of problems
\renewcommand{\homeworkProblemName}{#1} % Assign \homeworkProblemName the name of the problem
\section{\homeworkProblemName} % Make a section in the document with the custom problem count
\enterProblemHeader{\homeworkProblemName} % Header and footer within the environment
}{
\exitProblemHeader{\homeworkProblemName} % Header and footer after the environment
}

\newcommand{\problemAnswer}[1]{ % Defines the problem answer command with the content as the only argument
\noindent\framebox[\columnwidth][c]{\begin{minipage}{0.98\columnwidth}#1\end{minipage}} % Makes the box around the problem answer and puts the content inside
}

\newcommand{\homeworkSectionName}{}
\newenvironment{homeworkSection}[1]{ % New environment for sections within homework problems, takes 1 argument - the name of the section
\renewcommand{\homeworkSectionName}{#1} % Assign \homeworkSectionName to the name of the section from the environment argument
\subsection{\homeworkSectionName} % Make a subsection with the custom name of the subsection
\enterProblemHeader{\homeworkProblemName\ [\homeworkSectionName]} % Header and footer within the environment
}{
\enterProblemHeader{\homeworkProblemName} % Header and footer after the environment
}

%----------------------------------------------------------------------------------------
%	EXPECTATION AND VARIANCE OPERATOR
%----------------------------------------------------------------------------------------
 \newcommand{\E}{\mathrm{E}}
 \newcommand{\Var}{\mathrm{Var}}
 \newcommand{\Cov}{\mathrm{Cov}}
 \newcommand{\Corr}{\mathrm{Corr}}

%----------------------------------------------------------------------------------------
%	NAME AND CLASS SECTION
%----------------------------------------------------------------------------------------

\newcommand{\hmwkTitle}{MIL Exercise} % Assignment title
\newcommand{\hmwkDueDate}{Friday,\ March\ 30,\ 2018} % Due date
\newcommand{\hmwkClass}{FIN\ 521} % Course/class
\newcommand{\hmwkClassTime}{2:00pm} % Class/lecture time
\newcommand{\hmwkAuthorName}{Wanbae Park} % Your name

%----------------------------------------------------------------------------------------
%	TITLE PAGE
%----------------------------------------------------------------------------------------

\title{
\vspace{2in}
\textmd{\textbf{\hmwkClass:\ \hmwkTitle}}\\
\normalsize\vspace{0.1in}\small{Due\ on\ \hmwkDueDate}\\
\vspace{3in}
}

\author{\textbf{\hmwkAuthorName}}
\date{} % Insert date here if you want it to appear below your name

%----------------------------------------------------------------------------------------

\begin{document}

\maketitle

%----------------------------------------------------------------------------------------
%	TABLE OF CONTENTS
%----------------------------------------------------------------------------------------

%\setcounter{tocdepth}{1} % Uncomment this line if you don't want subsections listed in the ToC

%%\newpage
%%\tableofcontents
\newpage

%----------------------------------------------------------------------------------------
%	PART 1
%----------------------------------------------------------------------------------------
\begin{homeworkProblem}
    I chose PEPSICO INC. which is famous multinational food, snack, and
    beverage corporation. Most of data used in evaluating share price is come
    from CapitalIQ, and if data which is necessary to evaluate stock price
    is unobserved, it is estimated by myself.
    \begin{enumerate}[a)]
        \item   %% question a)
        I collected analyst's forecast of sales revenue of PepsiCo over next
        seven years. Table \ref{tab:part1-expectedcf} shows the estimated
        revenue. Estimated revenue grows moderately.
        \item   %% question b)
        Analyst's forecast data of company's EBITDA from FY2018 to FY2022
        is collected from CapitalIQ, and since the average EBITDA growth is
        about 6\% per year, I assumed that EBITDA will grow at 6\% over the
        last 2 years. Table \ref{tab:part1-expectedcf} shows the estimated
        EBITDA. Using this data, EBITDA/Sales margin was calculated.
        Table \ref{tab:part1-margins} shows estimated EBITDA/Sales margin.
        \item   %% question c)
        In order to estimate free cash flow, estimation for EBIT is needed first.
        As mentioned above, EBITDA was estimated by using analyst's forecast
        and by estimation of growth. Depreciation and amortization is estimated
        by estimating depreciation margin which is (Depreciation / Sales Revenue).
        Depreciation margin is estimated by averaging past 3 years' margin, and
        is assumed constant over 7 years. The margin was calculatd as 3.44\%.
        Tax rate used for calculating tax affected EBIT is 21\%.
        Changes in net working capital is estimated by estimating days in
        receivables, inventory, payables and others. They were estimated by
        averaging past 3 years' data and were assumed to be constant over 7 years.
        Table \ref{tab:part1-estimated days} shows estimated days of each account.
        Capital expenditure is estimated by estimating CapEx margin, which was
        calculated by averaging past 3 year's data. From the procedure, free
        cash flow was calculated as Table \ref{tab:part1-expectedcf}.
        \begin{table}[ht]
\centering
\begin{tabular}{@{}cccccccc@{}}
\toprule
                           & FY2018  & FY2019  & FY2020  & FY2021  & FY2022  & FY2023  & FY2024  \\ \midrule
Revenues                   & 65,373  & 67,472  & 69,619  & 71,229  & 73,403  & 77,621  & 80,373  \\
Gross Profit               & 35,782  & 36,931  & 38,105  & 38,987  & 40,176  & 42,485  & 43,992  \\
EBITDA                     & 13,627  & 14,402  & 15,129  & 16,286  & 17,188  & 18,219  & 19,312  \\
Depreciation/Amortization  & 2,248   & 2,321   & 2,394   & 2,450   & 2,525   & 2,670   & 2,764   \\
Operating Income           & 11,378  & 12,081  & 12,735  & 13,836  & 14,663  & 15,550  & 16,548  \\
Taxes                      & 2,389   & 2,537   & 2,674   & 2,906   & 3,079   & 3,265   & 3,475   \\
Tax Affected EBIT          & 8,989   & 9,544   & 10,060  & 10,931  & 11,584  & 12,284  & 13,073  \\ \midrule
Depreciation/Amortization  & 2,248   & 2,321   & 2,394   & 2,450   & 2,525   & 2,670   & 2,764   \\
Changes in Working Capital & 382     & (6)     & (6)     & (5)     & (6)     & (12)    & (8)     \\
Capital Expenditures       & (3,026) & (3,124) & (3,223) & (3,298) & (3,398) & (3,593) & (3,721) \\ \midrule
Unlevered Free Cash Flow   & 8,593   & 8,735   & 9,226   & 10,078  & 10,704  & 11,348  & 12,109  \\ \bottomrule
\end{tabular}
\caption{Estimated Income Statement and Expected Cash Flow}
\label{tab:part1-expectedcf}
\end{table}

        \begin{table}[ht]
\centering
\begin{tabular}{@{}cccccccc@{}}
\toprule
                    & FY2018  & FY2019  & FY2020  & FY2021  & FY2022  & FY2023  & FY2024  \\ \midrule
Sales Growth        & 2.91\%  & 3.21\%  & 3.18\%  & 2.31\%  & 3.05\%  & 5.75\%  & 3.55\%  \\
Gross Margin        & 54.73\% & 54.73\% & 54.73\% & 54.73\% & 54.73\% & 54.73\% & 54.73\% \\
EBITDA Margin       & 20.84\% & 21.34\% & 21.73\% & 22.86\% & 23.42\% & 23.47\% & 24.03\% \\
Depreciation Margin & 3.44\%  & 3.44\%  & 3.44\%  & 3.44\%  & 3.44\%  & 3.44\%  & 3.44\%  \\
CapEx Margin        & 4.63\%  & 4.63\%  & 4.63\%  & 4.63\%  & 4.63\%  & 4.63\%  & 4.63\%  \\ \bottomrule
\end{tabular}
\caption{Estimated Margins}
\label{tab:part1-margins}
\end{table}

        \begin{table}[ht]
\centering
\begin{tabular}{@{}ccccc@{}}
\toprule
                  & FY2015 & FY2016 & FY2017 & Average \\ \midrule
Days in AR        & 37     & 39     & 40     & 39      \\
Days in Inventory & 35     & 35     & 37     & 36      \\
Prepaid Days      & 14     & 11     & 19     & 19      \\
Days in AP        & 70     & 80     & 85     & 85      \\
Days in Accruals  & 52     & 55     & 53     & 53      \\ \bottomrule
\end{tabular}
\caption{Estimated days}
\label{tab:part1-estimated days}
\end{table}

        \item   %% question d)
        Since the food, snack and beverage industry is matured enough,
        it does not seem that the company will grow fast. Therefore,it is
        reasonable to assume that long run growth rate of free cash flow will be
        stable. Therefore, I chose the appropriate long run growth rate as 2\%.
        \item   %% question e)
        In order to calculate WACC of PepsiCo, cost of equity and cost of debt
        was calculated first. Using CapitalIQ's data, equity beta of the company
        is estimated as about 0.7. Since the given risk-free rate and market risk
        premium is 2\% and 5\%, respectively, using CAPM, cost of equity of
        the firm is calculated as $2\% + 0.7 \times 5\% = 5.53\%$. (Equity beta
        is slightly larger than 0.7 in actual.) Since the credit rating of the
        firm is A+, debt beta is estimated as 0.05. Then by using CAPM, cost
        of debt is calculated as 2.25\%. Applying tax benefit to cost of debt,
        after tax cost of debt is calculated as 1.78\%. From the balance sheet
        of the firm, proportion of equity financing is about 80.27\%, and
        proportion of debt financing is about 19.73\%. Using this data, WACC
        of the firm is calculated as 4.79\%.
        \item   %% question f)
        Using the long run growth rate and WACC calculated in d) and e), by
        using perpetuity formula, terminal value is calculated as 443,148.(million dollars)
        Discounted terminal value is calculatd as 319,444, and present value
        of estimated free cash flow is calculated as about 58,306. Therefore,
        enterprise value of the firm is estimated as $319,444 + 58,306 = 377,750$.
        \item   %% question g)
        From the balance sheet of the firm, total amount of debt is 39,281
        and amount of cash is 10,610. Therefore, equity value is calculated as
        $377,750 - 39,281 + 10,610 = 349,079$. Considering there are 1,420(million)
        shares outstanding, the share price is calculated as \$245.85 per share.
        \item   %% question h)
        Table \ref{tab:part1-sensitivity analysis} shows sensitivity analysis
        by changing EBITDA margins, long run growth rate, equity beta and
        market risk premium. It can be found that share price is very sensitive
        to equity beta. Of course, price is also sensitive to other factors, but
        considering that estimation of equity beta is more difficult than other
        factors, it is hard to say that the estimated share price is reliable.
        \begin{table}[ht]
\centering
\begin{tabular}{@{}cccccccc@{}}
\toprule
EBITDA Margin & Price  & Equity Beta & Price  & Risk Premium & Price  & Growth Rate & Price  \\ \midrule
-5\%          & 177.59 & 0.2         & 985.91 & 2.5\%               & 546.13 & 0.01                 & 184.82 \\
-4\%          & 191.88 & 0.3         & 635.13 & 3.0\%               & 445.67 & 0.015                & 210.69 \\
-3\%          & 206.18 & 0.4         & 464.98 & 3.5\%               & 375.24 & 0.02                 & 245.85 \\
-2\%          & 220.47 & 0.5         & 364.51 & 4.0\%               & 323.13 & 0.025                & 296.38 \\
-1\%          & 234.76 & 0.6         & 298.22 & 4.5\%               & 283.02 & 0.03                 & 375.18 \\
0\%           & 245.85 & 0.7         & 245.85 & 5.0\%               & 245.85 &                      &        \\
1\%           & 263.34 & 0.8         & 216.11 & 5.5\%               & 225.32 &                      &        \\
2\%           & 277.64 & 0.9         & 188.94 & 6.0\%               & 203.89 &                      &        \\
3\%           & 291.93 & 1           & 167.27 & 6.5\%               & 185.83 &                      &        \\
4\%           & 306.22 & 1.1         & 149.60 & 7.0\%               & 170.42 &                      &        \\
5\%           & 320.51 & 1.2         & 134.91 & 7.5\%               & 157.11 &                      &        \\ \bottomrule
\end{tabular}
\caption{Sensitivity Analysis}
\label{tab:part1-sensitivity analysis}
\end{table}

    \end{enumerate}
\end{homeworkProblem}

%----------------------------------------------------------------------------------------
%	PART 2
%----------------------------------------------------------------------------------------
\begin{homeworkProblem}
    In order to use comps method, I chose three firms, Coca Cola Company,
    Dr. Pepper Snapple Group, and Monster Beverage Corp. All of three are
    American beverage company like PepsiCo.
    \begin{enumerate}[a)]
        \item   %% question a)
        In order to value the firm, the average P/E ratio of comparable
        companies is used. Table \ref{tab:part2-p/e ratio} represents P/E ratio
        of comparable firms.
        \begin{table}[ht]
\centering
\begin{tabular}{@{}ccccc@{}}
\toprule
           & Coca Cola & Dr.Pepper & Monster & Average \\ \midrule
Price      & 42.89     & 118.28    & 56.25   & 72.47   \\
P/E Ratio  & 38.92     & 27.73     & 39.61   & 35.42   \\ \bottomrule
\end{tabular}
\caption{P/E Ratio of Comparable Firms}
\label{tab:part2-p/e ratio}
\end{table}

        Considering PepsiCo's net income and number of outstanding shares are
        \$4,857(million), and 1,420 million shares, respectively, using the average
        P/E ratio, price of PepsiCo per share is estimated as $4,857 \times 35.42
        / 1,420 = 121.16$.
        \item   %% question b)
        In this case, EV/EBITDA is used for evaluating PepsiCo's share price.
        Table \ref{tab:part2-ev/ebitda} represents EV/EBITDA of compariable firms.
        \begin{table}[ht]
\centering
\begin{tabular}{@{}ccccc@{}}
\toprule
          & Coca Cola & Dr.Pepper & Monster & Average \\ \midrule
EV        & 211035    & 25711     & 31010   & 89252   \\
EBITDA    & 10870     & 1579      & 1283    & 4577    \\
EV/EBITDA & 19.41     & 16.28     & 24.17   & 19.96   \\ \bottomrule
\end{tabular}
\caption{EV/EBITDA of Comparable Firms}
\label{tab:part2-ev/ebitda}
\end{table}

        Since PepsiCo's current EBITDA is 12,723, estimated enterprise value is
        $12,723 \times 19.96 = 253,891.4$. Considering total amount debt is equal
        to 39,281 and there is 10,610 amount of cash, equity value is calculated
        as $253,891.4 - 39,821 + 10,610 = 225,220.4$. Dividing equity value by
        number of shares(1,420), price per share is estimated as 158.62.
    \end{enumerate}
\end{homeworkProblem}

%----------------------------------------------------------------------------------------
%	PART 3
%----------------------------------------------------------------------------------------
\begin{homeworkProblem}
    In this case, comps method is more appropriate to evaluate share price than
    using DCF method. Although DCF method has more rigorous theoretical background,
    it might less reliable because implementation of the method is difficult because
    estimation of future cash flow and cost of capital is difficult. As shown in
    sensitivity analysis, share price of the firm is sensitive of lots of parameters
    which is not guaranteed to be estimated correctly. Of course, comps method
    also has some caveats like comparable companies choosing issue. However,
    as long as appropriate companies are found, evaluating share price is much easier
    and more accurate. Since all of comparable companies used in this case have
    similar industrial background, they might have similar multiples as PepsiCo's.
    Therefore, comps method might be more appropriate to evaluate PepsiCo's price.
    Since the current stock price of Pepsi is \$112.81, it can be found that
    evaluated price in all methods is higher than market price. Therefore, we
    can conclude that the market price of Pepsi is underpriced.
\end{homeworkProblem}
\end{document}
