%%%%%%%%%%%%%%%%%%%%%%%%%%%%%%%%%%%%%%%%%
% Structured General Purpose Assignment
% LaTeX Template
%
% This template has been downloaded from:
% http://www.latextemplates.com
%
% Original author:
% Ted Pavlic (http://www.tedpavlic.com)
%
% Note:
% The \lipsum[#] commands throughout this template generate dummy text
% to fill the template out. These commands should all be removed when
% writing assignment content.
%
%%%%%%%%%%%%%%%%%%%%%%%%%%%%%%%%%%%%%%%%%

%----------------------------------------------------------------------------------------
%	PACKAGES AND OTHER DOCUMENT CONFIGURATIONS
%----------------------------------------------------------------------------------------

\documentclass{article}

\usepackage{fancyhdr} % Required for custom headers
\usepackage{lastpage} % Required to determine the last page for the footer
\usepackage{extramarks} % Required for headers and footers
\usepackage{graphicx} % Required to insert images
\usepackage{lipsum} % Used for inserting dummy 'Lorem ipsum' text into the template
\usepackage{enumerate}
\usepackage{booktabs}
\usepackage{amsmath}
\usepackage{booktabs}
\usepackage{multirow}

% Margins
\topmargin=-0.45in
\evensidemargin=0in
\oddsidemargin=0in
\textwidth=6.5in
\textheight=9.0in
\headsep=0.25in

\linespread{1.5} % Line spacing

% Set up the header and footer
\pagestyle{fancy}
\lhead{\hmwkAuthorName} % Top left header
\chead{\hmwkClass\ (\hmwkTitle)} % Top center header
%%\rhead{\firstxmark}
\rhead{} % Top right header
\lfoot{\lastxmark} % Bottom left footer
\cfoot{} % Bottom center footer
\rfoot{Page\ \thepage\ of\ \pageref{LastPage}} % Bottom right footer
\renewcommand\headrulewidth{0.4pt} % Size of the header rule
\renewcommand\footrulewidth{0.4pt} % Size of the footer rule

\setlength\parindent{0pt} % Removes all indentation from paragraphs

%----------------------------------------------------------------------------------------
%	DOCUMENT STRUCTURE COMMANDS
%	Skip this unless you know what you're doing
%----------------------------------------------------------------------------------------

% Header and footer for when a page split occurs within a problem environment
\newcommand{\enterProblemHeader}[1]{
\nobreak\extramarks{#1}{#1 continued on next page\ldots}\nobreak
\nobreak\extramarks{#1 (continued)}{#1 continued on next page\ldots}\nobreak
}

% Header and footer for when a page split occurs between problem environments
\newcommand{\exitProblemHeader}[1]{
\nobreak\extramarks{#1 (continued)}{#1 continued on next page\ldots}\nobreak
\nobreak\extramarks{#1}{}\nobreak
}

\setcounter{secnumdepth}{0} % Removes default section numbers
\newcounter{homeworkProblemCounter} % Creates a counter to keep track of the number of problems

\newcommand{\homeworkProblemName}{}
\newenvironment{homeworkProblem}[1][Question \arabic{homeworkProblemCounter}]{ % Makes a new environment called homeworkProblem which takes 1 argument (custom name) but the default is "Question #"
\stepcounter{homeworkProblemCounter} % Increase counter for number of problems
\renewcommand{\homeworkProblemName}{#1} % Assign \homeworkProblemName the name of the problem
\section{\homeworkProblemName} % Make a section in the document with the custom problem count
\enterProblemHeader{\homeworkProblemName} % Header and footer within the environment
}{
\exitProblemHeader{\homeworkProblemName} % Header and footer after the environment
}

\newcommand{\problemAnswer}[1]{ % Defines the problem answer command with the content as the only argument
\noindent\framebox[\columnwidth][c]{\begin{minipage}{0.98\columnwidth}#1\end{minipage}} % Makes the box around the problem answer and puts the content inside
}

\newcommand{\homeworkSectionName}{}
\newenvironment{homeworkSection}[1]{ % New environment for sections within homework problems, takes 1 argument - the name of the section
\renewcommand{\homeworkSectionName}{#1} % Assign \homeworkSectionName to the name of the section from the environment argument
\subsection{\homeworkSectionName} % Make a subsection with the custom name of the subsection
\enterProblemHeader{\homeworkProblemName\ [\homeworkSectionName]} % Header and footer within the environment
}{
\enterProblemHeader{\homeworkProblemName} % Header and footer after the environment
}

%----------------------------------------------------------------------------------------
%	EXPECTATION AND VARIANCE OPERATOR
%----------------------------------------------------------------------------------------
 \newcommand{\E}{\mathrm{E}}
 \newcommand{\Var}{\mathrm{Var}}
 \newcommand{\Cov}{\mathrm{Cov}}
 \newcommand{\Corr}{\mathrm{Corr}}

%----------------------------------------------------------------------------------------
%	NAME AND CLASS SECTION
%----------------------------------------------------------------------------------------

\newcommand{\hmwkTitle}{Problem Set\ \#3} % Assignment title
\newcommand{\hmwkDueDate}{Wednesday,\ April\ 11,\ 2018} % Due date
\newcommand{\hmwkClass}{FIN\ 521} % Course/class
\newcommand{\hmwkClassTime}{2:00pm} % Class/lecture time
\newcommand{\hmwkAuthorName}{Wanbae Park} % Your name

%----------------------------------------------------------------------------------------
%	TITLE PAGE
%----------------------------------------------------------------------------------------

\title{
\vspace{2in}
\textmd{\textbf{\hmwkClass:\ \hmwkTitle}}\\
\normalsize\vspace{0.1in}\small{Due\ on\ \hmwkDueDate}\\
\vspace{3in}
}

\author{\textbf{\hmwkAuthorName}}
\date{} % Insert date here if you want it to appear below your name

%----------------------------------------------------------------------------------------

\begin{document}

\maketitle

%----------------------------------------------------------------------------------------
%	TABLE OF CONTENTS
%----------------------------------------------------------------------------------------

%\setcounter{tocdepth}{1} % Uncomment this line if you don't want subsections listed in the ToC

%%\newpage
%%\tableofcontents
\newpage

%----------------------------------------------------------------------------------------
%	QUESTION 1
%----------------------------------------------------------------------------------------
\begin{homeworkProblem}
    \begin{enumerate}[a.]
        \item   %% a.
        Since the risk is idiosyncratic, discount rate of the firm is 0.
        Because all the possibilities of future value is equally likely,
        Gladstone's current equity value is equal to
        $0.25 \times (150 + 135 + 95 + 80) = 115$ million dollars.
        \item   %% b.
        If the firm earns \$150 or \$135 million in the future, debt holder
        would get 100\% of face value. However, if the firm earns \$95 or
        \$80 million dollars in the future, debt holder would get \$95,
        \$80 million dollars, respectively. Since risk is diversifiable,
        current value of the debt is equal to
        $0.25 \times 100 + 100 + 95 + 80 = 93.75$.
        \item   %% c.
        Yield is a discount rate affected to ``promised" payoff. Expected return,
        in contrast, is a discount rate affected to ``expected" payoff.
        Therefore, yield of the debt satisfies $\frac{100}{1 + y} = 93.75$.
        Therefore $y = \frac{100 - 93.75}{93.75} = 6.67\%$. In order to
        calculated expected return, it needs to calculate expected payoff of
        the debt, from b. it is calculated as 93.75. Therefore, in this case,
        expected return is equal to 0. It is because there is no systematic risk.
        \item   %% d.
        From a, expected present value of the future value is equal to 115.
        Since debt value is equal to 93.75, equity value is equal to 115 - 93.75
        = 21.25, and firm's total value is 115, which is same as a. Therefore,
        Modigliani \& Miller I holds here.
    \end{enumerate}
\end{homeworkProblem}

%----------------------------------------------------------------------------------------
%	QUESTION 2
%----------------------------------------------------------------------------------------
\begin{homeworkProblem}
    \begin{enumerate}[a.]
        \item   %% a.
        Since the firm is unlevered, cost of capital of the firm is equal to
        unlevered cost of capital, which is 15\%. Therefore, the firm value
        (present value of future free cash flow) is equal to as follows.
        \begin{equation*}
            \begin{aligned}
                PV(FCF) &= \frac{5}{1.15} + \frac{5}{(1.15)^2}
                + \frac{5}{(1.15)^3} + \dots    \\
                        &= \frac{5}{0.15} = 33.33
            \end{aligned}
        \end{equation*}
        \item   %% b.
        In this case, we can calculate firm's value by adding present value
        of income tax shield to unlevered firm's value. Since the debt lasts
        permanently, present value of interest is same as current debt value.
        Therefore, interest tax shield is equal to
        (tax rate) $\times$ (current debt value), so the firm value is equal
        to $33.33 + 0.35 \times 19.05 = 40$.
    \end{enumerate}
\end{homeworkProblem}

%----------------------------------------------------------------------------------------
%	QUESTION 3
%----------------------------------------------------------------------------------------
\begin{homeworkProblem}
    \begin{enumerate}[a.]
    \item   %% a.
    Before the announcement, since Kurz has 20 million outstanding shares and
    price per share is \$7.50, total value of the firm is equal to
    $7.50 \times 20 = 150$ million dollars because it is all-equity firm.
    \item   %% b.
    Since investors know the effect of income tax shield, total value of the
    firm will be increased after the announcement, and tha amount of increase
    is equal to amount of income tax shield.
    Assuming there is no excess cash in Kurz, because the debt is expected
    to be maintained permanently, the present value of income tax shield will
    be equal to $0.40 \times 50 = 20$ million dollars. Therefore, firm value
    will be increased to 170 million dollars, and it is all-equity.
    \item   %% c.
    Since firm value is increased to 170 million dollars and there are 20
    million outstanding shares, price per shere increases to $170 / 20 = 8.5$.
    Therefore, Kurz will purchase about $50 / 8.5 = 5.89$ million shares.
    \item   %% d.
    Since 50 million amount of shares are repurchased, equity value will
    decrease by 50 million, so equity value will be 120 million after repurchase.
    Since number of outstanding share also decreased, price per share is not
    changed. The price is $\frac{120}{20 - 5.89} = 8.5$, which is same as
    the price before repurchase.
    \end{enumerate}
\end{homeworkProblem}

%----------------------------------------------------------------------------------------
%	QUESTION 4
%----------------------------------------------------------------------------------------
\begin{homeworkProblem}
    \begin{enumerate}[a.]
    \item   %% a.
    Table \ref{tab:prob4-2mil} shows income tax shield and cost of financial distress
    and NPV, when cost of financial distress is equal to 2 million.
    In this case, the optimal debt is equal to \$80 million because NPV is the
    largest.
    \begin{table}[ht]
\centering
\begin{tabular}{@{}cccccccc@{}}
\toprule
Debt Level                          & 0     & 40    & 50    & 60    & 70    & 80    & 90    \\ \midrule
PV of Income Tax Shield             & 0     & 0.76  & 0.95  & 1.14  & 1.33  & 1.52  & 1.71  \\
Probability of Financial Distress   & 0     & 0     & 0.01  & 0.02  & 0.07  & 0.16  & 0.31  \\
Present Value of Financial Distress & 0.000 & 0.000 & 0.019 & 0.038 & 0.133 & 0.305 & 0.590 \\
NPV                                 & 0.000 & 0.760 & 0.931 & 1.102 & 1.197 & 1.215 & 1.120 \\ \bottomrule
\end{tabular}
\caption{Income tax shield and cost of financial distress: 2 million}
\label{tab:prob4-2mil}
\end{table}

    \item   %% b.
    Table \ref{tab:prob4-5mil} shows income tax shield and cost of financial distress
    and NPV, when cost of financial distress is equal to 5 million.
    In this case, the optimal debt is equal to \$60 million.
    \begin{table}[ht]
\centering
\begin{tabular}{@{}cccccccc@{}}
\toprule
Debt Level                          & 0     & 40    & 50    & 60    & 70    & 80    & 90    \\ \midrule
PV of Income Tax Shield             & 0     & 0.76  & 0.95  & 1.14  & 1.33  & 1.52  & 1.71  \\
Probability of Financial Distress   & 0     & 0     & 0.01  & 0.02  & 0.07  & 0.16  & 0.31  \\
Present Value of Financial Distress & 0.000 & 0.000 & 0.048 & 0.095 & 0.333 & 0.762 & 1.476 \\
NPV                                 & 0.000 & 0.760 & 0.902 & 1.045 & 0.997 & 0.758 & 0.234 \\ \bottomrule
\end{tabular}
\caption{Income tax shield and cost of financial distress: 5 million}
\label{tab:prob4-5mil}
\end{table}

    \item   %% c.
    Table \ref{tab:prob4-25mil} shows income tax shield and cost of financial distress
    and NPV, when cost of financial distress is equal to 25 million.
    In this case, the optimal debt is equal to \$40 million.
    From a, b, c, we can find that optimal debt decreases as cost of financial
    distress increases.
    \begin{table}[ht]
\centering
\begin{tabular}{@{}cccccccc@{}}
\toprule
Debt Level                          & 0     & 40    & 50    & 60    & 70     & 80     & 90     \\ \midrule
PV of Income Tax Shield             & 0     & 0.76  & 0.95  & 1.14  & 1.33   & 1.52   & 1.71   \\
Probability of Financial Distress   & 0     & 0     & 0.01  & 0.02  & 0.07   & 0.16   & 0.31   \\
Present Value of Financial Distress & 0.000 & 0.000 & 0.238 & 0.476 & 1.667  & 3.810  & 7.381  \\
NPV                                 & 0.000 & 0.760 & 0.712 & 0.664 & -0.337 & -2.290 & -5.671 \\ \bottomrule
\end{tabular}
\caption{Income tax shield and cost of financial distress: 25 million}
\label{tab:prob4-25mil}
\end{table}

    \end{enumerate}
\end{homeworkProblem}

%----------------------------------------------------------------------------------------
%	QUESTION 5
%----------------------------------------------------------------------------------------
\begin{homeworkProblem}
    \begin{enumerate}[a.]
    \item   %% a.
    Before paying dividend, since there are 500 million shares and current
    share price is \$15, market value of equity is equal to \$7,500 million.
    After paying dividend, equity value decreases to \$7,250 million, and
    since the number of share does not change, price per share decreases to
    $\frac{7,250}{500} = \$14.5$.
    \item   %% b.
    If the board decided to repurchase the same amount of share, the number
    of share the firm will buy is equal to $250 / 15 = 16.7$ shares.
    After repurchase, equity value of firm decreases, and number of outstanding
    share also decreases. Therefore, price per share is equal to
    $\frac{7,250}{500 - 16.7} = 15$, which is unchanged.
    \item   %% c.
    Firm's investors are indifferent to both decisions. From decision (a),
    amount of value for investor is 7,500, 7,250 is for equity, and the rest
    is for cash flow from dividend. Similarly, if the firm repurchase shares,
    value for investors is same because they get 250 million dollars for
    repurchase, and have 7,250 million dollars in equity.
    \end{enumerate}
\end{homeworkProblem}

%----------------------------------------------------------------------------------------
%	QUESTION 6
%----------------------------------------------------------------------------------------
\begin{homeworkProblem}

\end{homeworkProblem}

%----------------------------------------------------------------------------------------
%	QUESTION 7
%----------------------------------------------------------------------------------------
\begin{homeworkProblem}
    \begin{enumerate}[a.]
    \item   %% a.
    \item   %% b.
    \end{enumerate}
\end{homeworkProblem}

%----------------------------------------------------------------------------------------
%	QUESTION 8
%----------------------------------------------------------------------------------------
\begin{homeworkProblem}
    \begin{enumerate}[a.]
    \item   %% a.
    \item   %% b.
    \end{enumerate}
\end{homeworkProblem}
\end{document}
