%%%%%%%%%%%%%%%%%%%%%%%%%%%%%%%%%%%%%%%%%
% Structured General Purpose Assignment
% LaTeX Template
%
% This template has been downloaded from:
% http://www.latextemplates.com
%
% Original author:
% Ted Pavlic (http://www.tedpavlic.com)
%
% Note:
% The \lipsum[#] commands throughout this template generate dummy text
% to fill the template out. These commands should all be removed when
% writing assignment content.
%
%%%%%%%%%%%%%%%%%%%%%%%%%%%%%%%%%%%%%%%%%

%----------------------------------------------------------------------------------------
%	PACKAGES AND OTHER DOCUMENT CONFIGURATIONS
%----------------------------------------------------------------------------------------

\documentclass{article}

\usepackage{fancyhdr} % Required for custom headers
\usepackage{lastpage} % Required to determine the last page for the footer
\usepackage{extramarks} % Required for headers and footers
\usepackage{graphicx} % Required to insert images
\usepackage{lipsum} % Used for inserting dummy 'Lorem ipsum' text into the template
\usepackage{enumerate}
\usepackage{booktabs}
\usepackage{amsmath}
\usepackage{booktabs}
\usepackage{multirow}

% Margins
\topmargin=-0.45in
\evensidemargin=0in
\oddsidemargin=0in
\textwidth=6.5in
\textheight=9.0in
\headsep=0.25in

\linespread{1.5} % Line spacing

% Set up the header and footer
\pagestyle{fancy}
\lhead{\hmwkAuthorName} % Top left header
\chead{\hmwkClass\ (\hmwkTitle)} % Top center header
%%\rhead{\firstxmark}
\rhead{} % Top right header
\lfoot{\lastxmark} % Bottom left footer
\cfoot{} % Bottom center footer
\rfoot{Page\ \thepage\ of\ \pageref{LastPage}} % Bottom right footer
\renewcommand\headrulewidth{0.4pt} % Size of the header rule
\renewcommand\footrulewidth{0.4pt} % Size of the footer rule

\setlength\parindent{0pt} % Removes all indentation from paragraphs

%----------------------------------------------------------------------------------------
%	DOCUMENT STRUCTURE COMMANDS
%	Skip this unless you know what you're doing
%----------------------------------------------------------------------------------------

% Header and footer for when a page split occurs within a problem environment
\newcommand{\enterProblemHeader}[1]{
\nobreak\extramarks{#1}{#1 continued on next page\ldots}\nobreak
\nobreak\extramarks{#1 (continued)}{#1 continued on next page\ldots}\nobreak
}

% Header and footer for when a page split occurs between problem environments
\newcommand{\exitProblemHeader}[1]{
\nobreak\extramarks{#1 (continued)}{#1 continued on next page\ldots}\nobreak
\nobreak\extramarks{#1}{}\nobreak
}

\setcounter{secnumdepth}{0} % Removes default section numbers
\newcounter{homeworkProblemCounter} % Creates a counter to keep track of the number of problems

\newcommand{\homeworkProblemName}{}
\newenvironment{homeworkProblem}[1][Question \arabic{homeworkProblemCounter}]{ % Makes a new environment called homeworkProblem which takes 1 argument (custom name) but the default is "Question #"
\stepcounter{homeworkProblemCounter} % Increase counter for number of problems
\renewcommand{\homeworkProblemName}{#1} % Assign \homeworkProblemName the name of the problem
\section{\homeworkProblemName} % Make a section in the document with the custom problem count
\enterProblemHeader{\homeworkProblemName} % Header and footer within the environment
}{
\exitProblemHeader{\homeworkProblemName} % Header and footer after the environment
}

\newcommand{\problemAnswer}[1]{ % Defines the problem answer command with the content as the only argument
\noindent\framebox[\columnwidth][c]{\begin{minipage}{0.98\columnwidth}#1\end{minipage}} % Makes the box around the problem answer and puts the content inside
}

\newcommand{\homeworkSectionName}{}
\newenvironment{homeworkSection}[1]{ % New environment for sections within homework problems, takes 1 argument - the name of the section
\renewcommand{\homeworkSectionName}{#1} % Assign \homeworkSectionName to the name of the section from the environment argument
\subsection{\homeworkSectionName} % Make a subsection with the custom name of the subsection
\enterProblemHeader{\homeworkProblemName\ [\homeworkSectionName]} % Header and footer within the environment
}{
\enterProblemHeader{\homeworkProblemName} % Header and footer after the environment
}

%----------------------------------------------------------------------------------------
%	EXPECTATION AND VARIANCE OPERATOR
%----------------------------------------------------------------------------------------
 \newcommand{\E}{\mathrm{E}}
 \newcommand{\Var}{\mathrm{Var}}
 \newcommand{\Cov}{\mathrm{Cov}}
 \newcommand{\Corr}{\mathrm{Corr}}

%----------------------------------------------------------------------------------------
%	NAME AND CLASS SECTION
%----------------------------------------------------------------------------------------

\newcommand{\hmwkTitle}{Problem Set\ \#4} % Assignment title
\newcommand{\hmwkDueDate}{Sunday,\ May\ 6,\ 2018} % Due date
\newcommand{\hmwkClass}{FIN\ 521} % Course/class
\newcommand{\hmwkClassTime}{2:00pm} % Class/lecture time
\newcommand{\hmwkAuthorName}{Wanbae Park} % Your name

%----------------------------------------------------------------------------------------
%	TITLE PAGE
%----------------------------------------------------------------------------------------

\title{
\vspace{2in}
\textmd{\textbf{\hmwkClass:\ \hmwkTitle}}\\
\normalsize\vspace{0.1in}\small{Due\ on\ \hmwkDueDate}\\
\vspace{3in}
}

\author{\textbf{\hmwkAuthorName}}
\date{} % Insert date here if you want it to appear below your name

%----------------------------------------------------------------------------------------

\begin{document}

\maketitle

%----------------------------------------------------------------------------------------
%	TABLE OF CONTENTS
%----------------------------------------------------------------------------------------

%\setcounter{tocdepth}{1} % Uncomment this line if you don't want subsections listed in the ToC

%%\newpage
%%\tableofcontents
\newpage

%----------------------------------------------------------------------------------------
%	QUESTION 1
%----------------------------------------------------------------------------------------
\begin{homeworkProblem}
    \begin{enumerate}[a.]
        \item   %% a.
        Let $x$ denote the current price per share. Then market capitalization
        of the firm is equal to $x \times$ 8 million dollars before the investment.
        After investment, market capitalization of the firm will be $8x + 1$
        million dollars, and the portion of venture capitalist will be
        $\frac{1}{8x + 1}$, which will be equal to 0.2.
        Therefore, by solving the equation, current price of share is equal to
        0.5 dollars, and the venture capitalist will get 2 million shares.
        \item   %% b.
        Since there are 10 million shares after investent and price per share
        is \$0.5, the value of firm is equal to $0.5 \times 2 = 1$ million dollars.
    \end{enumerate}
\end{homeworkProblem}

%----------------------------------------------------------------------------------------
%	QUESTION 2
%----------------------------------------------------------------------------------------
\begin{homeworkProblem}
    \begin{enumerate}[a.]
        \item   %% a.
        Since the IPO price was \$20 per share and there is 7\% underwriting
        spread, the amount of capital raise is equal to
        $(1 - 0.07) \times 20 \times 5 = 93$ million dollars.
        \item   %% b.
        After IPO, since 5 million shares are added, there are 15 million shares
        of the firm. Because the share price increased to \$50, market value of
        equity of the firm is equal to $50 \times 15 = 750$ million dollars.
        \item   %% c.
        Since market is perfect, the current share price of firm must be equal
        to \$50. Therefore, because there is 10 million shares before issuing
        stock, pre-money value of the equity is equal to $50 \times 10 = 500$
        million dollars. Under perfect market, because the firm will issue stock
        at the fair price: \$50, if the firm issues 5 million shares, the amount
        of capital risen is equal to $50 \times 5 = 250$ million dollars.
        Therefore, post-money value of equity is equal to 750 million dollars.
        Under this circumstances, in order to raise 93 million dollars as in
        question a, it needs to issue $93 / 50 = 1.86$ million shares, which
        is quite less than the amount of issuance at question a.
        \item   %% d.
        Due to underpricing and underwriting spread, the firm can only raise
        93 million dollars for issuing 5 million shares, comparing 250 million
        dollars when market is perfect. Therefore, it can be concluded that
        $250 - 93 = 157$ million dollars are left on the table due to market
        imperfection.
    \end{enumerate}
\end{homeworkProblem}

%----------------------------------------------------------------------------------------
%	QUESTION 3
%----------------------------------------------------------------------------------------
\begin{homeworkProblem}
    \begin{enumerate}[a.]
        \item   %% a.
        \item   %% b.
    \end{enumerate}
\end{homeworkProblem}

%----------------------------------------------------------------------------------------
%	QUESTION 4
%----------------------------------------------------------------------------------------
\begin{homeworkProblem}
    \begin{enumerate}[a.]
        \item   %% a.
        \item   %% b.
        \item   %% c.
        \item   %% d.
    \end{enumerate}
\end{homeworkProblem}

%----------------------------------------------------------------------------------------
%	QUESTION 5
%----------------------------------------------------------------------------------------
\begin{homeworkProblem}

\end{homeworkProblem}


%----------------------------------------------------------------------------------------
%	QUESTION 6
%----------------------------------------------------------------------------------------
\begin{homeworkProblem}
    \begin{enumerate}[a.]
        \item   %% a.
        \item   %% b.
    \end{enumerate}
\end{homeworkProblem}


%----------------------------------------------------------------------------------------
%	QUESTION 7
%----------------------------------------------------------------------------------------
\begin{homeworkProblem}

\end{homeworkProblem}


%----------------------------------------------------------------------------------------
%	QUESTION 8
%----------------------------------------------------------------------------------------
\begin{homeworkProblem}

\end{homeworkProblem}

\end{document}
